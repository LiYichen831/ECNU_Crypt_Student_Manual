\documentclass{article}
\usepackage{amsmath, amssymb, amsthm}
\usepackage{geometry}
\usepackage{ctex}
\geometry{margin=1in}

\title{区块链基础第一次作业}
\author{10225101408 朱炎}
\date{2025.3.18}

\begin{document}

\maketitle

比特币总量上限为2100万个且预计在2140年基本挖完的核心机制源于其区块奖励减半规则。

\section{比特币的总量上限计算}
比特币的初始奖励为50个比特币,每210000个区块奖励减半一次,第一个周期内产生$50 \times 210000 = 10500000$个比特币。
则我们可以通过以下等比数列求和公式计算比特币总量上限:
\begin{equation}
    10500000 \times \sum_{n=0}^{\infty} (\frac{1}{2})^n = 10500000 \times \frac{1}{1 - \frac{1}{2}} = 21000000 \text{BTC}
\end{equation}
即2100万个比特币。

\section{比特币的终止时间计算}
比特币每210000个区块奖励减半一次,每个区块的产生时间约为10分钟,故210000个区块的生成耗时约为210000*10分钟 = 35000小时 = 1458天 = 4年。

比特币于2009年1月3日开始发行,到2140年约为131年,即32个周期。32个周期后,比特币的奖励为:
\begin{equation}
    50 \times 2^{-32} \approx 1.16 \times 10^{-8} \text{BTC}
\end{equation}
而比特币的最小单位1聪(Satoshi)为$10^{-8}$比特币,故比特币的奖励将约等于0,挖矿终止。

\end{document}
